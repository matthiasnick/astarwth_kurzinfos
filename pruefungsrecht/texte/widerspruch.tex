\begin{artikel}{Widerspruchsverfahren}
Prüfungsentscheidungen sind Verwaltungsakte. Wichtigstes Rechtsmittel gegen sie ist der Widerspruch. Mit diesem wendest Du dich an die zuständige Widerspruchsbehörde, in der Regel an Deinen Prüfungsausschuss. Der Widerspruch muss schriftlich und begründet erfolgen. In der Regel beträgt die Frist hierfür einen Monat. Sie verlängert sich auf ein Jahr, wenn keine oder eine fehlerhafte Rechtsbehelfsbelehrung erfolgt ist. Der Prüfungsausschuss entscheidet zeitnah über den Widerspruch. Wenn Dein Widerspruch nicht erfolgreich ist, kannst Du Anfechtungsklage vor dem Verwaltungsgericht erheben. Wenn Du nicht weißt, wie oder an wen Du Deinen Widerspruch richten musst, hilft Dir Deine Fachschaft oder der AStA.
\end{artikel}
