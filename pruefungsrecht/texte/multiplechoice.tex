\begin{artikel}{FAQ Multiple Choice}
Das Antwort-Wahl-Verfahren, häufig Multiple-Choice genannt, ist rechtlich besonders umstritten. Gemeint sind Aufgaben, bei denen aus vorgegebenen Antwortmöglichkeiten richtige Antworten ausgewählt werden müssen. Dabei gibt es die unterschiedlichsten Formen und Bewertungsarten. Das Verfahren ist zwar einfach in der Auswertung, dafür fehlt aber die Möglichkeit, sinnvoll mit fehlerhaften oder mehrdeutigen Fragestellungen umzugehen. Die Bewertung kann keine Rücksicht auf die Richtigkeit von Antworten nehmen, die von der Erwartung der Prüferin oder des Prüfers abweichen. Daher gibt es hohe Anforderungen an das Stellen der Aufgaben.

\textit{Der Abzug von Punkten für falsch angekreuzte oder fehlerhaft nicht angekreuzte Antworten ist nach neuester Rechtssprechung im Allgemeinen rechtswidrig, da so darüber hinweggetäuscht wird, dass richtige Antworten gewusst wurden. Das bedeutet aber nicht, dass es genügt, alle Antwortmöglichkeiten anzukreuzen. Es darf lediglich für eine falsch beantwortete Aufgabe insgsamt keinen Abzug von Punkten geben.}

Bei Aufgaben im Antwort-Wahl-Verfahren muss es immer eine relative Bestehensgrenze geben, d.h. die Gesamtleistung aller Teilnehmenden an einer Prüfung muss berücksichtigt werden. \textit{Insbesondere ist eine Prüfung im Antwort-Wahl-Verfahren immer als bestanden zu bewerten, wenn 60\% der richtigen Antworten gegeben worden sind oder wenn die Anzahl der korrekt gegebenen Antworten höchstens 22\% unter dem Durchschnitt derer liegt, die zum ersten Mal an der Prüfung teilnehmen.}
\end{artikel}
