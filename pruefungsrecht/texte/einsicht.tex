\begin{artikel}{FAQ Einsichtnahme}
Nach jeder vollständig abgelegten Prüfung steht dir das Recht auf Einsichtnahme in die Prüfungsakten zu, unabhängig davon, ob du eine Prüfung bestanden hast oder nicht. Du musst die Bewertung nachvollziehen können. Die Einsicht nur unter Aufsicht zu erlauben, ist zulässig. Auf jeden Fall muss man dir ausreichend Zeit und Platz zur Verfügung zu stellen. \textit{Obwohl dies häufig verweigert wird, hast du das Recht, dir Notizen zu machen und Auszüge zu fertigen. Es muss zur Begründung eines Widerspruchs sogar ermöglicht werden, Kopien anzufertigen oder anfertigen zu lassen, die Kosten dafür hast du selbst zu tragen.}
\end{artikel}
