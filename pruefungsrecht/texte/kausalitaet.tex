\begin{artikel}{Kausalität}
Im Allgemeinen kann eine Prüfungsentscheidung aufgehoben werden, wenn gegen Verfahrensvorschriften verstoßen wurde. Wurde also ein Fehler bei Deinem Prüfungsverfahren gemacht, kann dies ein Anrecht darauf begründen, die Prüfung zu wiederholen. Das gilt aber aber nur, wenn ein Einfluss des Verfahrensfehlers auf das Ergebnis nicht ausgeschlossen werden kann. Hättest Du zum Beispiel eine Prüfung auch dann nicht bestanden, wenn Du bei einer fehlerhaft gestellten Aufgabe alle Punkte erreicht hättest, so kann der Fehler keine Auswirkung auf Dein Bestehen gehabt haben.

Die Beweislast dafür, dass ein Verfahrensfehler keine Auswirkungen hatte, liegt in der Regel bei der Prüfungsbehörde.
\end{artikel}
