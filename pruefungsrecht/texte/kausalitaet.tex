\begin{artikel}{Kauslität}
Nur weil gegen Verfahrensvorschriften verstoßen wurde, muss die Prüfungsentscheidung nicht unbedingt aufgehoben werden. Wurde also ein Fehler bei deinem Prüfungsverfahren gemacht, hast du nicht immer einen Anspruch darauf, dass du die Prüfung wiederholen kannst. Grundsätzlich kannst du davon ausgehen, aber nur solange ein Einfluss des Verfahrensfehlers auf das Ergebnis nicht ausgeschlossen werden kann. Hättest du zum Beispiel eine Prüfung auch dann nicht bestanden, wenn du bei einer fehlerhaft gestellten Aufgabe alle Punkte erreicht hättest, so kann der Fehler keine Auswirkung auf dein Bestehen gehabt haben.

Die Beweislast dafür, dass ein Verfahrensfehler keine Auswirkungen hatte, liegt in der Regel bei der Prüfungsbehörde.
\end{artikel}
