\begin{artikel}{Deine Prüfungsordnung}
Hochschulprüfungen werden aufgrund von Prüfungsordnungen abgelegt. Das sind Satzungen, die Vorgaben durch Gesetze, Grundrechte und Rechtssprechung unterliegen. Zunächst ist wichtig, dass Du Deine Prüfungsordnung gelesen hast. Deine aktuell gültige Prüfungsordnung findest Du online in den Amtlichen Bekanntmachungen der RWTH. Achte hierbei auch auf Änderungsordnungen. Prüfungsordnungen legen u.a. die abzulegenden Prüfungsleistungen sowie ihre Art und Dauer fest, ebenso die Möglichkeiten der An- und Abmeldung und Wiederholungsmöglichkeiten.

2009 hat der Senat der RWTH Rahmenprüfungsordnungen für alle Bachelor- und Masterstudiengänge (außer Lehramt) erlassen, um die Bedingungen für alle Studierenden vergleichbar zu machen. Genauere Informationen findest Du in der Kurzinfo zu den RahmenPOs.

Neben den in den Prüfungsordnungen festgelegten Bedingungen hat die RWTH ihre internen Abläufe in einem Wiki zusammengefasst. Dieses erreichst Du aus dem RWTH-Netz unter \url{https://wiki-intern.rwth-aachen.de}.
\end{artikel}
