\begin{artikel}{Kosten}
Beim Widerspruchsverfahren entstehen normalerweise nur Kosten, wenn anwaltliche Hilfe in Anspruch genommen wird. Die Kosten für eine Anfechtungsklage und das Eilverfahren werden im Wesentlichen durch die Anwaltskosten bestimmt, die vom gewählten Rechtsanwalt, Streitwert, Aufwand etc. abhängen. Normalerweise ist auch die Beratung kostenpflichtig, eine kostenlose Beratung für alle Studierenden der RWTH gibt es regelmäßig im AStA.

Viele Studierende haben z.B. über die Eltern eine Rechtsschutzversicherung. Diese decken häufig Verwaltungsrecht und insbesondere Prüfungsrecht leider nicht ab.
\end{artikel}
