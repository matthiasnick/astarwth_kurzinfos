\begin{artikel}{Kosten}
Beim Widerspruchsverfahren entstehen normalerweise nur Kosten, wenn anwaltliche Hilfe in Anspruch genommen wird. Die Kosten für eine Anfechtungsklage und das Eilverfahren werden im Wesentlichen durch die Anwaltskosten bestimmt, die vom gewählten Rechtsanwalt, Streitwert, Aufwand etc. abhängen. Normalerweise ist auch die Beratung kostenpflichtig, eine kostenlose Beratung für alle Studierenden der RWTH gibt es regelmäßig im AStA.

Beachte, dass eine möglicherweise vorhandene Rechtsschutzversicherung unter Umständen Verwaltungsrecht und insbesondere Prüfungsrecht nicht abdeckt.
\end{artikel}
