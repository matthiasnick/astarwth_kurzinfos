\begin{artikel}{FAQ Krankheit, Rücktritt und Versäumnis}
Die Möglichkeiten zum Prüfungsrücktritt werden durch die Prüfungsordnung geregelt. Die meisten Ordnungen ermöglichen neben einer ein- oder mehrmaligen Prüfungsabmeldung den Prüfungsrücktritt aus wichtigem Grund. Der häufigste Grund ist die Prüfungsunfähigkeit wegen Krankheit. Diese muss in der Regel vor Beginn der Prüfung angezeigt werden und die Prüfung darf nicht angetreten werden. Es ist nicht zulässig, mit dem Attest in der Tasche zur Prüfung zu gehen und sich hinterher auf Prüfungsunfähigkeit zu berufen.
Wichtig ist, dass ein Rücktritt von einer Prüfung erklärt, also bekannt gegeben, wird. Wenn du einfach nicht zur Prüfung erscheinst, wird die Prüfung grundsätzlich als "`nicht ausreichend"' gewertet.
Es gibt in Einzelfällen die Möglichkeit, dass Du Deine Prüfungsunfähigkeit erst während oder gar nach der Prüfung feststellst. In diesen Fällen gelten weitergehende Anforderungen an das Attest. Näheres hierzu findest du im "'Wiki Intern"`.
\end{artikel}
