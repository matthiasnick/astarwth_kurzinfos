\begin{artikel}{FAQ Wiederholungsprüfungen und letzter Versuch}
Die meisten Prüfungsordnungen sehen eine zweimalige Wiederholungsmöglichkeit jeder nicht bestandenen Prüfung vor, manche eine mündliche Ergänzungsprüfung. Wiederholungsprüfungen müssen dabei nicht zwangsläufig nach denselben Verfahrensregeln wie die erste Prüfung durchgeführt werden.

Wenn du eine oder mehrere Prüfungen nicht bestanden hast und den Studiengang oder die Hochschule wechseln möchtest, kann das mit Schwierigkeiten verbunden sein. Lass dich in so einem Fall beraten.

Prüfungsleistungen in Prüfungen, mit denen ein Studiengang abgeschlossen wird, und in Wiederholungsprüfungen, bei deren endgültigen Nichtbestehen keine Ausgleichsmöglichkeit vorgesehen ist, sind von mindestens zwei Prüferinnen oder Prüfern zu bewerten.

Nach gängiger Rechtsprechung darf die Zweitkorrektur aber in Kenntnis der Erstkorrektur und -bewertung vorgenommen werden und die Erstkorrektur bestätigt werden.

Ein Sonderfall ist die Bewertung von Aufgaben im Antwort-Wahl-Verfahren (Multiple-Choice). Das Zusammenzählen der Punkte ist keine Bewertung. Die Prüfertätigkeit ist auf das Stellen der Aufgaben vorverlagert. Daher müssen nach neuester Rechtsprechung bei der Erstellung der Aufgaben zwei Prüfer zusammenwirken.
\end{artikel}
