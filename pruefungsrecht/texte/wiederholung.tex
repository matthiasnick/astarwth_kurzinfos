\begin{artikel}{FAQ Wiederholungsprüfungen und letzter Versuch}
Die meisten Prüfungsordnungen sehen eine zweimalige Wiederholungsmöglichkeit jeder nicht bestandenen Prüfung sowie eine mündliche Ergänzungsprüfung vor. Wiederholungsprüfungen müssen dabei nicht zwangsläufig nach denselben Verfahrensregeln wie die erste Prüfung durchgeführt werden.

Wenn Du eine oder mehrere Prüfungen nicht bestanden hast und den Studiengang oder die Hochschule wechseln möchtest, können sich Schwierigkeiten ergeben. Lass Dich in so einem Fall beraten.

Leistungen in Prüfungen, mit denen ein Studiengang abgeschlossen wird oder bei deren endgültigen Nichtbestehen keine Ausgleichsmöglichkeit vorgesehen ist, sind von mindestens zwei Prüferinnen oder Prüfern zu bewerten. Nach gängiger Rechtsprechung darf die Zweitkorrektur aber in Kenntnis der Erstkorrektur und -bewertung vorgenommen werden und dabei die Erstkorrektur bestätigt werden.

Ein Sonderfall ist die Bewertung von Aufgaben im Antwort-Wahl-Verfahren (Multiple-Choice). Das Zusammenzählen der Punkte ist keine Bewertung. Die Prüfertätigkeit ist auf das Stellen der Aufgaben vorverlagert. Daher müssen nach aktueller Rechtsprechung bei der Erstellung der Aufgaben zwei Prüferinnen bzw. Prüfer zusammenwirken.
\end{artikel}
