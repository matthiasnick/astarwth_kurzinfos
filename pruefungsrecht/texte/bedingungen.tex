\begin{artikel}{FAQ Äußere Prüfungsbedingungen, Prüfungsdauer und Rügepflicht}
Die äußeren Prüfungsbedingungen müssen aus Gründen der Chancengleichheit für alle Geprüften identisch sein. Gelegentlich treten im Prüfungsablauf aber auch Störungen wie Lärm, Hitze, Kälte oder beißender Geruch auf. Überschreiten diese Störung eine Erheblichkeitsschwelle, können sie Dein Leistungsvermögen beeinträchtigen.

Dies kannst Du nach der Prüfung nur geltend machen, wenn Du es unverzüglich bei der Aufsicht oder der Prüferin bzw. dem Prüfer rügst und um Abhilfe bittest. Die Rüge solltest Du im Protokoll festhalten lassen. Wenn sofortige Abhilfe nicht möglich ist, kann eine Schreibzeitverlängerung gewährt werden.

Die Prüfungsdauer wird in der Regel durch die Prüfungsordnung bestimmt und muss, abgesehen vom Fall der Schreibzeitverlängerung, eingehalten werden. Die Prüferin bzw. der Prüfer darf diese weder verkürzen noch verlängern, auch nicht wenn Du zustimmst.
\end{artikel}
