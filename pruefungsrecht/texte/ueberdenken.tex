\begin{artikel}{Verwaltungsinternes Überdenken}
Der Prüferin bzw. dem Prüfer steht ein gewisser Beurteilungsspielraum zu, innerhalb dessen eine gerichtliche Überprüfung nicht möglich ist.

Aus diesem Grund gibt es das verwaltungsinterne Überdenkungsverfahren, bei dem du der Prüferin bzw. dem Prüfer anhand von konkreten Einwänden die Möglichkeit geben musst, die Prüfungsentscheidung zu überdenken - in der Regel im Rahmen der Einsicht in die Prüfungsakten. Sofern es nicht offensichtliche Formfehler gegeben hat, wie zum Beispiel falsch zusammengezählte Punkte, ist ein Verschlechterung ausgeschlossen. Es werden nur die beanstandeten Einzelwertungen überprüft.
\end{artikel}
