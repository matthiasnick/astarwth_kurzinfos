\begin{artikel}{der Mietvertrag}
Die meisten Vermieter werden Dir einen Standardmietvertrag (z.B. vom Haus- und Grundbesitzerverein) vorlegen. Legt der Vermieter einen 
eigenen (oder modifizierten) Mietvertrag vor ist auf jeden Fall Vorsicht geboten. Den Vertrag ganz genau zu lesen sollte sowieso 
selbstverst�ndlich sein. Ist dir die Bedeutung von Teilen des Vertrages unklar so solltest du fachkundigen Rat einholen (z.B. im 
AStA-Wohnungsreferat).

Unterschreibe den Vertrag auf jeden Fall erst, wenn sichergestellt ist, dass Dir nicht irgendwelche Vertragsklauseln zum Verh�ngnis werden 
k�nnten!\\

\textbf{Vorsicht vor:}
\begin{enumerate}
\item Staffelmieten (Mieterh�hungen sind bereits im Mietvertrag festgeschrieben)
\item Zeitmietvertr�gen (hier bist du einen festen Zeitraum an den Vertrag gebunden und kannst vorher quasi nicht k�ndigen)
\item�bernahme von Wohnungsinventar (Pass auf, dass der Vormieter nicht seinen Sperrm�ll an dich verscherbelt)
\end{enumerate}
\textbf{Wichtig:}
\begin{enumerate}
\item Alle Vereinbarungen und Absprachen mit dem Vermieter schriftlich festhalten. M�ndliche Absprachen sind rechtlich zwar auch bindend 
aber im Zweifelsfall deutlich schwerer einzufordern.
\item Bei der Wohungs�bergabe ein �bergabeprotokoll anfertigen. Darin alle sichtbaren M�ngel (Feuchtigkeit, Flecken im Teppich oder auf 
der 
Tapete usw.) und �bergabezustand (frisch renoviert?, wenn ja wann?) schriftlich festhalten.
\item Lass dir die Nebnkostenabrechnung des Vorjahres vorlegen, damit keine hohen Nachzahlungen f�llig werden. Der Vorauszahlungsbetrag 
kann 
unter Umst�nden sehr weit von den tats�chlichen Nebenkosten abweichen.
\item Genau notieren welche Sch�den (falls vorhanden) bis wann vom Vermieter beseitigt werden. Die Frist sollte hier genau vermerkt und 
ihre 
Einhaltung im Zweifelsfall auch eingefordert werden.
\item F�r alle F�lle solltest Du jemanden mitnehmen, der als Zeuge aussagen kann.
\end{enumerate}
\end{artikel}
