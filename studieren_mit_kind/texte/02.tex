\begin{artikel}{Finanzielle Unterst"utzung}
Es gibt verschiedene M"oglichkeiten eine finanzielle Unterst"utzung, auch schon w"ahrend der Schwangerschaft, zu erhalten:
So besteht die M"oglichkeit die \textbf{Erstausstattung der Schwangerenbekleidung} bei der zust"andigen Arbeitsagentur zu beantragen (ca. 750 Euro;  WICHTIG: Quittungen aufbewahren!!!)
Zudem kann \textbf{Mutterschaftsgeld} 7 Wochen vor dem errechneten Geburtstermin bei der Krankenversicherung beantragt werden. Dieses wird ab 6 Wochen vor und bis 8 Wochen nach der Geburt gezahlt. Die H"ohe des Mutterschaftsgeldes kann bei der jeweiligen Krankenversicherung erfragt werden (ca.13 Euro pro Kalendertag).

Des Weiteren entsteht  mit der Geburt eines Kindes der Anspruch auf \textbf{Kindergeld}, das unabh"angig vom Einkommen der Eltern gezahlt wird. F"ur das erste und zweite Kind werden monatlich 184 Euro, f"ur das dritte Kind 190 Euro und f"ur jedes weitere Kind 215 Euro gezahlt. Das Kindergeld kannst Du schriftlich bei der Familienkasse, der f"ur deinen Wohnort zust"andigen Arbeitsagentur beantragen.

Zus"atzlich kannst Du auch einen \textbf{Kinderzuschlag} von 140 Euro pro Monat erhalten, sofern Du mit Deinen Eink"unften zwar den eigenen Unterhalt, aber nicht den Deiner Kinder finanzieren kannst. Der Antrag auf Kinderzuschlag kann ebenfalls bei der Familienkasse, der f"ur den Wohnort zust"andigen Agentur f"ur Arbeit, auch r"uckwirkend, gestellt werden.
Sofern Du \textbf{BAf"oG} bekommst, kannst Du einen \textbf{Zuschlag} in H"ohe von 175 Euro f"ur das erste Kind bzw. 85 Euro f"ur jedes weitere Kind erhalten. Des Weiteren sieht das BAf"oG f"ur studierende Eltern eine Verl"angerung der F"orderungsh"ochstdauer vor (WICHTIG: BAf"oG wird nicht w"ahrend eines Urlaubssemesters gezahlt).\\

Zus"atzlich kannst Du \textbf{Elterngeld} bei der zust"andigen Elterngeldstelle beantragen. Das Elterngeld wird an V"ater und M"utter f"ur maximal 14 Monate gezahlt (ein Elternteil kann h"ochstens 12 Monate Elterngeld in Anspruch nehmen); beide k"onnen den Zeitraum frei untereinander aufteilen. Die Bedingungen daf"ur sind, dass das Kind im Haushalt lebt und selbst erzogen und betreut wird und Du nicht mehr als 30 Stunden in der Woche erwerbst"atig bist.  Als pauschale Mindestsumme werden 300 Euro Elterngeld je Kind  pro Monat an Studierende gew"ahrt. 
Des Weiteren besteht die M"oglichkeit \textbf{Wohngeld}\footnote{N"ahere Informationen zum Thema Wohngeld erh"alst du auf der Wohngeldstelle der Stadt Aachen, oder bei der Internetpr"asenz der Stadt  Aachen.} bei der Stadt zu beantragen. Dieses betr"agt ca. 140 Euro.


Es gibt weitere Leistungen, die von der Krankenkasse "ubernommen werden. So k"onnen verschiedene Kosten im Rahmen von Schwangerschaft und Geburt "ubernommen werden. Manche  Krankenkassen "ubernehmen auch die Kosten f"ur die Schwangerschaftsgymnastik. "Uber die verschiedenen Leistungen kannst Du Dich bei Deiner Krankenkasse informieren.
\end{artikel}
