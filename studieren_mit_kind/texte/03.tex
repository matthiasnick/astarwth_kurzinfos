\begin{artikel}{Weitere wichtige Informationen:}
Die Schwangerschaft muss nicht bei der Hochschule gemeldet werden, au"ser Du hast einen Arbeitsvertrag oder musst ein Praktikum f"ur das Studium absolvieren. Des Weiteren tritt der Mutterschutz (6 Wochen vor dem Geburtstermin bis 8 Wochen nach der Geburt) nur in Kraft, sofern du dich in einem Arbeitsverh"altnis oder Praktikum befindest.
Urlaubssemester k"onnen ebenfalls im Fall einer Schwangerschaft, allerdings erst ab dem zweiten Fachsemester, wahrgenommen werden. Ein "arztlicher Nachweis, dass eine Schwangerschaft vorliegt, muss zus"atzlich vorgelegt werden. Den Antrag auf Beurlaubung kannst Du pers"onlich bei Deinem*Deiner Ansprechpartner*in im Studierendensekretariat stellen. Eltern k"onnen bis zu 6 Urlaubssemester an der Hochschule beantragen. M"ochten sich allerdings beide Elternteile zur gleichen Zeit beurlauben lassen, m"ussen sie diese sechs Semester unter sich aufteilen. (Wichtig: W"ahrend eines Urlaubssemesters wird kein BAf"oG gezahlt).
\end{artikel}

