\begin{artikel}{Der Bescheid}

Wenn das BAf"oG-Amt Deinen Antrag bearbeitet hat, bekommst Du den sogenannten BAf"oG Bescheid zugesendet. In diesem Bescheid ist aufgef"uhrt, wie viel Geld Du im kommenden Jahr monatlich an BAf"oG bekommst. Au"serdem kannst Du sehen, wie viel Geld Du laut BAf"oG Gesetz pro Monat brauchst und wie viel Dir Deine Eltern geben m"ussen.

\frage{Was mache ich, wenn ich glaube, dass sich das BAf"oG Amt bei der Berechnung vertan hat?}
Solltest Du das Gef"uhl haben, dass etwas mit Deinem Bescheid nicht stimmt, such m"oglichst schnell eine Beratungsstelle auf und vergewissere Dich, ob die Berechnung richtig ist. Du solltest Dich damit beeilen, da Du nur einen Monat Zeit hast gegen den BAf"oG Bescheid Widerspruch einzulegen.

\frage{Was mache ich, wenn meine Eltern nicht bereit sind ihren Anteil zu zahlen?}
In einem solchen Fall musst Du dem Amt f"ur Ausbildungsf"orderung glaubhaft machen, dass Deine Eltern nicht bereit sind, den Anteil zu "ubernehmen, den sie zu Deinem Bedarf beisteuern m"ussen. Dann ist es m"oglich, eine BAf"oG-Vorausleistung zu erhalten. Du bekommst dann das gesamte ben"otigte Geld. Dabei gehen Deine Unterhaltsanspr"uche auf das BAf"oG-Amt "uber, das sich wiederum bem"uhen wird, das Geld von Deinen Eltern einzutreiben. Dieses Verfahren ist meistens mit der Anh"orung Deiner Eltern verbunden und hat f"ur sie rechtliche Konsequenzen. Du solltest nat"urlich alles versuchen, um dieses Problem mit Deinen Eltern anderweitig zu kl"aren. In schwierigeren F"allen hilft es vielleicht, Sie auf obige Regelung hinzuweisen.
\end{artikel}

