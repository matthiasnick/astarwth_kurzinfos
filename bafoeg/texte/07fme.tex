\vspace{-0.1cm}
\begin{artikel}{Die 5 h"aufigsten Fehler beim Erstantrag}
\vspace{-0.6cm}
\begin{enumerate}
\item Den Antrag zu sp"at abgeben.
\item Sich nicht erkundigen, ob von Verwandten Geld auf Deinen Namen angelegt wurde, von dem Du nichts wei"st. (Dieses wird in jedem Fall Dir zugerechnet und gilt bei einem etwaigen Datenabgleich als nicht angegebenes Verm"ogen.)
\item Den ausgef"ullten Erstantrag mitten im ersten Semester abgeben, ohne zum Anfang des Semesters einen formlosen Antrag gestellt zu haben. (Du verlierst Deine Anspr"uche vom Anfang des Semesters bis zum Eingang Deines Antrags).
\item Sich bei kritischen Fragen nicht beraten lassen. Insbesondere bei der Einkommensaktualisierung.
\item Bei einem "`eigenartigen"' Bescheid keine Beratung einholen. (Nach einem Monat kannst Du formal gesehen nicht mehr widersprechen und der Bescheid ist f"ur ein Jahr f"ur Dich rechtsverbindlich).
\end{enumerate}
Linkliste/Weitere Informationen
\begin{itemize}
\item \href{http://www.bafoeg.bmbf.de}{www.bafoeg.bmbf.de} - Bundesministerium f"ur Bildung und Forschung (incl. BAf"oG-Rechner)
\item \href{http://www.studentenwerk-aachen.de}{www.studentenwerk-aachen.de} - BAf"oG Amt Aachen
\item \href{http://www.studis-online.de}{www.studis-online.de} und \href{http://www.bafoegrechner.de}{www.bafoegrechner.de} aktuelle Infos rund ums BAf"oG
\end{itemize}
Der AStA bietet auch eine pers"onliche Beratung an. Die genauen Termine findest Du in der unten stehenden Kontaktbox oder auf unserer Webseite unter \href{http://www.asta.rwth-aachen.de/beratung}{www.asta.rwth-aachen.de/beratung}.\\

Eine vorherige Terminabsprache ist nicht notwendig. Die Beratung ist kostenlos und erfolgt selbstverst"andlich vertraulich.
\end{artikel}

