\begin{artikel}{}
\vspace{-20pt}
beginn ein Bescheid in Deinem Briefkasten sein. Vorausgesetzt nat"urlich, Du hast auch alle Nachweise fr"uhzeitig eingereicht.

Falls Du zwei Wochen danach noch keinen Bescheid hast, solltest Du Dich beim BAf"oG-Amt melden. Gibst Du Deinen Antrag erst wenige Wochen vor Semesterbeginn ab, musst Du damit rechnen eventuell mehrere Monate warten zu m"ussen. Du erh"altst dann eine Nachzahlung. Das BAf"oG-Amt gibt sich zwar bei Erstantr"agen alle M"uhe, schnell zu bescheiden. Wenn aber viele Antr"age sehr kurzfristig gestellt werden, dauert es nat"urlich recht lange.

\frage{Ich werde den Erstantrag samt Nachweisen nicht rechtzeitig abgeben k"onnen. Was muss ich beachten?}
Falls Du es nicht mehr bis Ende des ersten Studienmonats schaffst: Stell in jedem Fall einen formlosen Antrag beim BAf"oG-Amt, um Deinen Anspruch ab Studienbeginn nicht zu verlieren. Es reicht ein Brief mit dem Inhalt:

"`Hiermit beantrage ich, \{Vorname Nachname\}, Leistungen nach dem BAf"oG."'

Kannst Du auch die Frist zur Abgabe der restlichen Unterlagen nicht einhalten, die Dir vom BAf"oG-Amt auf Deinen "`Kurzantrag"' hin gesetzt wurde, sprich dies vorher mit dem Amt ab.
\end{artikel}
