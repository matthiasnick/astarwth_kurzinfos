\begin{artikel}{Die 5 h"aufigsten Fehler beim Erstantrag}
\noindent
\begin{enumerate}
%\setlength{\topsep}{0ex}
%\setlength{\labelwidth}{0ex}
\setlength{\parsep}{0mm}
\setlength{\itemsep}{0ex}
\setlength{\leftmargin}{-2ex}

\item  Den Antrag zu sp"at abgeben.
\item Sich nicht erkundigen, ob von Verwandten Geld auf Deinen Namen angelegt wurde, von dem Du nichts wei"st. (Dieses wird in jedem Fall Dir zugerechnet und gilt bei einem etwaigen Datenabgleich als nicht angegebenes Verm"ogen.)
\item Den ausgef"ullten Erstantrag mitten im ersten Semester abgeben, ohne zum Anfang des Semesters einen formlosen Antrag gestellt zu haben. (Du verlierst Deine Anspr"uche vom Anfang des Semesters bis zum Eingang eines Antrags).
\item Sich bei kritischen Fragen nicht beraten lassen. Insbesondere bei der Einkommensaktualisierung.
\item Bei einem "`eigenartigen"' Bescheid keine Beratung einholen. (Nach einem Monat kannst Du formal gesehen nicht mehr widersprechen und der Bescheid ist f"ur ein Jahr f"ur Dich rechtsverbindlich.). 
\end{enumerate}

\boxzt{Ansprechpersonen bei Problemen}
{
\vspace*{-.5ex}

\textbf{AStA der RWTH Aachen}\\
Turmstra"se 3\\
52072 Aachen\\
Tel.: 0241 80-93792\\
\url{www.asta.rwth-aachen.de}\\
eMail: \url{bafoeg@asta.rwth-aachen.de}
\vspace*{1ex}

\centerline{\includegraphics[width=.7\linewidth]{bilder/wir}}
\vspace*{1ex}

\textbf{Amt f"ur Ausbildungsf"orderung}\\
Turmstra"se 3\\
52072 Aachen\\
\url{www.studentenwerk-aachen.de}
}
\end{artikel}
