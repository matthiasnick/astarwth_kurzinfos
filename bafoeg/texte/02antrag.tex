\begin{artikel}{Antragstellung}

\frage{Was geh"ort in den Erstantrag?}
F"ur die Antragstellung musst Du einige Antragsformulare ausf"ullen. Diese Antr"age findest Du im Internet unter: \url{http://www.studentenwerk-aachen.de/bafoeg/bafoeg/download/start.asp} oder beim BAf"oG Amt in der Peterstra"se 44 - 46 in Aachen.

Alle BAf"oG-Empf"angerInnen m"ussen das Formblatt 1 und das Zusatzblatt zum Formblatt 1 ausf"ullen. Solltest Du elternabh"angiges BAf"oG bekommen (was der Regelfall ist), musst Du f"ur Deine Eltern das Formblatt 3 plus den Lohnsteuerbescheid von vor zwei Jahren abgeben. Sollten Deine Eltern nicht gemeinsam zur Lohnsteuer veranlagt werden, muss das Formblatt pro Elternteil mit Einkommen einmal ausgef"ullt werden. Wenn Du eine eigene Wohnung hast, musst Du noch eine Mietbescheinigung einreichen. Au"ser den Formbl"attern geh"ort noch ein Krankenversicherungsnachweis und eine Studienbescheinigung in den Antrag.

Falls Du keine deutsche Staatsangeh"origkeit hast, musst Du zus"atzlich noch das Formblatt 4 abgeben.

\frage{Wann sollte ich meinen Erstantrag abgeben?}
Am besten ca. drei Monate vor Beginn des ersten Semesters. Damit stellst Du sicher, dass "uber Deinen Antrag rechtzeitig entschieden wird, um zum Anfang des ersten Semesters F"orderung zu erhalten.

Sp"atestens musst Du den Antrag bis zum Ende des ersten Studienmonats stellen. Allerdings ist davon stark abzuraten, schlie"slich kann die Bearbeitung Deines Antrags eine ganze Weile dauern.

Du kannst auch noch nach Ende des ersten Studienmonats einen Antrag stellen, hast aber dann keinen Anspruch mehr auf eine Nachzahlung f"ur die vergangenen Monate.

\frage{Wie lange dauert die Entscheidung "uber meinen Antrag im Normalfall?}
Hast Du Deinen Antrag fr"uhzeitig abgegeben (ca. 3 Monate vor Beginn des ersten Semesters), sollte sp"atestens zum Semester-
\end{artikel}
