\begin{artikel}{}
\vspace{-16pt}
Neben der Entscheidung, welches Fach Du an welcher Hochschule studieren m"ochtest, stellt sich zum Studienanfang nat"urlich auch die Frage der Studienfinanzierung. Dabei geht es im Wesentlichen um das Aufbringen der allgemeinen Lebenshaltungskosten, die sp"atestens jetzt, wo Du hoffentlich eine eigene Wohnung in Aachen gefunden hast, auch auf Dich zukommen.

F"ur Studierende, die w"ahrend des Studiums finanziell nicht von ihren Eltern unterst"utzt werden k"onnen, gibt es die Bundesausbildungsf"orderung -- kurz auch BAf"oG genannt (von BundesAusbildungsf"orderungsGesetz).\\

Dieses Info-Blatt soll einen "Uberblick und Antworten auf die wichtigsten Fragen geben, die sich viele ErstsemesterInnen stellen. Da das BAf"oG Gesetz sehr komplex ist und es viele Ausnahmeregelungen gibt, k"onnen wir hier nur die Standard-F"alle beschreiben. F"ur detailliertere Fragen oder f"ur Probleme bei der Beantragung steht Dir im AStA eine BAf"oG-Beratung zur Verf"ugung.

Au"serdem findest Du auf der R"uckseite eine Aufstellung der wichtigsten Anlaufstellen, die zu der Antragstellung, dem BAf"oG-Bescheid oder der R"uckzahlung informieren oder beraten k"onnen.
\end{artikel}
