% Achtung: Dies ist der komplette Text der spaeter auf verschiedenen Dateien aufgeteilt wurden. Dieser Text ist nicht mehr aktuell und hat reinen Archivcharakter.

\begin{artikel}{BAf"oG -- alles rund um die Antragstellung}
\begin{window}[0,r,\includegraphics[width=.2\linewidth]{bilder/logo},]
Neben der Entscheidung, welches Fach man an welcher Hochschule studieren m"ochte, stellt sich zum Studienanfang nat"urlich auch die Frage, wie man das Studium finanzieren soll. Dabei geht es im wesentlichen um das Aufbringen der allgemeinen Lebenshaltungskosten, die sp"atestens jetzt, wo Du hoffentlich eine eigene Wohnung in Aachen gefunden hast, auch auf Dich zukommen. F"ur die Menschen, die weder wohlhabende Eltern haben, noch ein Fach studieren, welches genug Zeit l"a"st, um nebenher zu arbeiten, gibt es die Bundesausbildungsf"orderung -- kurz auch BAf"oG genannt (von BundesAusbildungsf"orderungsGesetz).

Um Dich bei der BAf"oG-Beantragung zu unterst"utzen, gibt der AStA eine Info-Brosch"ure heraus, in der alles von der Antragstellung bis zu verschiedensten Ausnahmen und Sonderregelungen erkl"art wird. Dieses Info-Blatt soll Dir daher nur einen ersten "Uberblick und Antworten auf die dr"angendsten Fragen geben.

Wenn Du dieses Info-Blatt liest, solltest Du den Antrag eigentlich schon gestellt haben. Viele Menschen stellen den Antrag aber leider erst kurz vor Studienbeginn. Haupts"achlich f"ur diese Letzte-Minute-AntragstellerInnen haben wir die folgenden Informationen zusammengestellt. Es sollen aber auch Anlaufstellen genannt und Fragen beantwortet werden, wenn es Probleme mit dem Bescheid vom BAf"oG-Amt gab. Am besten kommst Du in solchen F"allen zur BAF"oG-Beratung im AStA oder wendest Dich an Dein Amt f"ur Ausbildungsf"orderung. Die Kontaktadressen findest Du auf der zweiten Seite.
\end{window}
\end{artikel}
\begin{artikel}{Antragstellung in letzter Minute}
\vspace*{-1ex}%Die frage schlucken

\frage{Wann h"atte ich meinen Antrag stellen sollen?}
Am besten ca. 3 Monate vor Studienbeginn. Damit h"attest Du sichergestellt, dass Dein Antrag rechtzeitig beschieden (also entschieden) worden w"are und Du zum Anfang des ersten Semesters F"orderung erhalten h"attest.

\frage{Wann muss ich den Erstantrag sp"atestens abgegeben?}
Kurz gesagt: Formell bis sp"atestens Ende Oktober. Allerdings ist davon nat"urlich stark abzuraten, schlie"slich kann die Bescheidung Deines Antrags eine ganze Weile dauern. Du kannst zwar sp"ater auch noch einen Erstantrag stellen, erh"altst dann aber die F"orderung nicht ab Studienbeginn sondern ab dem Zeitpunkt der Antragstellung.

\frage{Wie lange dauert die Entscheidung "uber meinen Antrag (Bescheidung) im Normalfall?}
Hast Du Deinen Antrag fr"uhzeitig abgegeben (ca. zu Juli 2004), solltest Du gegen Ende September einen Bescheid in den H"anden halten. Vorausgesetzt nat"urlich, Du hast auch alle Nachweise rechtzeitig eingereicht. Falls Du zum Vorlesungsbeginn noch keinen Bescheid hast, solltest Du Dich beim BAf"oG-Amt melden. Gibst Du Deinen Antrag erst im Laufe des Septembers ab, musst Du damit rechnen eventuell bis Ende November warten zu m"ussen. Das BAf"oG-Amt gibt sich zwar bei Erstantr"agen alle M"uhe, schnell zu bescheiden, wenn aber viele Antr"age erst kurz vor Semesterbeginn gestellt werden, dauert es nat"urlich recht lange. 

\frage{Ich werde den Erstantrag samt Nachweisen nicht rechtzeitig abgeben k"onnen. Was muss ich beachten? }
Falls Du es nicht mehr bis Ende Oktober schaffst: 

Stell in jedem Fall einen formlosen Antrag beim BAf"oG-Amt, um Deinen Anspruch f"ur Oktober nicht zu verlieren. Es reicht ein Brief mit dem Inhalt: \textit{"`Hiermit beantrage ich, \textit{\{Vorname Nachname\}}, Leistungen nach dem BAf"oG."'}. Kannst Du auch die Frist zur Abgabe der restlichen Unterlagen nicht einhalten, die Dir vom BAf"oG-Amt auf Deinen "`Kurzantrag"' hin gesetzt wurde, sprich dies vorher mit dem Amt ab.
\end{artikel}
\begin{artikel}{Der Bescheid}
\vspace*{-1ex}


\frage{Nach dem Bescheid, den ich erhalten habe, bekomme ich sehr wenig/kein BAf"oG, weil meine Eltern zu viel Einkommen haben. Meine Eltern sind aber nicht bereit, mich zu unterst"utzen, obwohl sie dies finanziell k"onnten. Was kann ich machen?}
Kannst Du dem Amt f"ur Ausbildungsf"orderung glaubhaft machen, dass Deine Eltern nicht bereit sind, den Anteil zu "ubernehmen, der Dir von Deinem Bedarf abgezogen wurde (sofern er sich auf das Einkommen Deiner Eltern bezieht), ist es m"oglich, f"ur die erste Zeit BAf"oG unter Vorbehalt zu erhalten. Dabei gehen Deine Unterhaltsanspr"uche auf das Land NRW "uber, welches sich widerum bem"uhen wird, das Geld von Deinen Eltern einzutreiben. Dieses Verfahren ist meistens mit der Anh"orung Deiner Eltern verbunden und hat f"ur sie auch rechtliche Konsequenzen. Du solltest nat"urlich alles versuchen, um dieses Problem mit Deinen Eltern anderweitig zu kl"aren. In schwierigeren F"allen hilft es vielleicht, Sie auf obige Regelung hinzuweisen. 

\vfill

\frage{Meine Eltern haben vor ein paar Jahren gut verdient. Das hat sich leider jetzt ge"andert und sie k"onnen mich finanziell nicht unterst"utzen. Das BAf"oG-Amt geht aber von ihrem Einkommen vor zwei Jahren aus, wodurch ich jetzt keine oder wenig F"orderung erhalte. Kann ich irgendwas tun?}
Ja, Du kannst in dieser Situation einen sogenannten Aktualisierungsantrag stellen. Das BAf"oG-Amt pr"uft dann, ob es zu diesem Zeitpunkt finanziell f"ur dich von Vorteil w"are, neu zu bescheiden und vom aktuellen Einkommen Deiner Eltern auszugehen. Du solltest Dich vorher aber in jedem Fall beraten lassen, da ein solcher Antrag f"ur Dich mit finanziellen und rechtlichen Risiken verbunden ist. 
Sollte beispielsweise einer Deiner Eltern pl"otzlich eine lukrative Stelle annehmen, kannst Du nicht verlangen, wieder von ihrem Einkommen von vor zwei Jahren auszugehen.
%Du kannst beispielsweise sp"ater nicht mehr verlangen, doch von dem Einkommen vor zwei Jahren auszugehen, falls einer Deiner Eltern pl"otzlich eine lukrative Stelle annimmt.

 In diesem seltenen Fall musst Du meist die F"orderung kurzfristig zur"uckzahlen.
\end{artikel}
\begin{artikel}{Hochschul- und Fachwechsel}
\vspace*{-1ex} 

\frage{Ich habe bereits an einer anderen Hochschule dasselbe Fach studiert und wechsle zum Anfang des n"achsten Semesters an die RWTH. Kann ich weiter BAf"oG bekommen?}
Du bekommst im Allgemeinen weiter BAf"oG. Nat"urlich musst Du dazu wie immer einen Folgeantrag stellen, falls Dein Bewilligungszeitraum abl"auft. Au"serdem wird in Bezug auf den "`Leistungsnachweis"', den Du zum 5. Fachsemester erbracht haben musst, keine R"ucksicht auf Hochschulwechsel genommen. Da Du mit Sicherheit Reibungsverluste hast, ist es sinnvoll, die Studienberatung in Anspruch zu nehmen, um zu kl"aren, wie Du den Nachweis rechtzeitig erbringen kannst. 

\frage{Zum Wechsel an die RWTH habe ich auch mein Studienfach/Nebenfach gewechselt. Was muss ich beachten, um weiter BAf"oG zu bekommen?}

Ein kompliziertes Thema, welches wir hier nur kurz anreissen k"onnen. Ein Fachwechsel ohne Verlust Deines BAf"oG-Anspruchs ist im Allgemeinen nur bis zum Ende des dritten Semesters m"oglich und muss begr"undet werden. Lass Dich dazu beraten; insbesondere wenn Du weiterhin auf Magister studierst, kann es sein, dass BAf"oG-rechtlich "uberhaupt kein Fachwechsel vorliegt.

\frage{Ich denke der Bescheid ist fehlerhaft. Was muss ich beachten?}
Versuch Deine F"ordersumme nochmals mit dem BAf"oG-Rechner nachzurechnen (siehe Linkliste). Bist Du immernoch unsicher, dann lass Dich beraten. Wenn sich das Problem nicht kurzfristig kl"aren l"asst, solltest du schriftlich Widerspruch gegen den Bescheid bei Deinem BAf"oG-Amt einlegen. Auf dem Bescheid findest du eine Rechtsbehelfsbelehrung in der Dir eine Frist zum einreichen eines Widerspruchs gesetzt wird. Bisher betr"agt diese Frist einen Monat nach Bekanntgabe. Sie ist unbedingt einzuhalten.
\end{artikel}
\begin{artikel}{Die 5 h"aufigsten Fehler beim Erstantrag}
\noindent
\begin{enumerate}
%\setlength{\topsep}{0ex}
%\setlength{\labelwidth}{0ex}
\setlength{\parsep}{0mm}
\setlength{\itemsep}{0ex}
\setlength{\leftmargin}{-2ex}

\item  Den Antrag zu sp"at abgeben.
\item Sich nicht erkundigen, ob von Verwandten Geld auf Deinen Namen angelegt wurde, von dem Du nichts wei"st. (Dieses wird in jedem Fall Dir zugerechnet und gilt bei einem etwaigen Datenabgleich als nicht angegebenes Verm"ogen.)
\item Den ausgef"ullten Erstantrag mitten im ersten Semester abgeben, ohne zum Anfang des Semesters einen formlosen Antrag gestellt zu haben. (Du verlierst Deine Anspr"uche vom Anfang des Semesters bis zum Eingang eines Antrags).
\item Sich bei kritischen Fragen nicht beraten lassen. Insbesondere bei der Einkommensaktualisierung.
\item Bei einem "`eigenartigen"' Bescheid keine Beratung einholen. (Nach einem Monat kannst Du formal gesehen nicht mehr widersprechen und der Bescheid ist f"ur ein Jahr f"ur Dich rechtsverbindlich.). 
\end{enumerate}

\boxzt{Ansprechpersonen bei Problemen}
{
\vspace*{-.5ex}

\textbf{AStA der RWTH Aachen}\\
Turmstra"se 3\\
52072 Aachen\\
Tel.: 0241 80-93792\\
\url{www.asta.rwth-aachen.de}\\
eMail: \url{bafoeg@asta.rwth-aachen.de}
\vspace*{1ex}

\centerline{\includegraphics[width=.7\linewidth]{bilder/wir}}
\vspace*{1ex}

\textbf{Amt f"ur Ausbildungsf"orderung}\\
Turmstra"se 3\\
52072 Aachen\\
\url{www.studentenwerk-aachen.de}
}
\end{artikel}
\begin{artikel}{Linkliste/Weitere Informationen}
\parbox{.1\linewidth}{\includegraphics[width=\linewidth]{bilder/uhr}}
\parbox{.9\linewidth}{
\begin{aufzaehlung}
\item \url{www.asta.rwth-aachen.de} -- Webseiten des AStA mit aktuellen Infos und Sprechzeiten der BAf"oG-Beratung
\item \url{www.bafoeg.bmbf.de} -- Bundesministerium f"ur Bildung und Forschung, Informationen und BAf"oG-Rechner
\item \url{www.studentenwerk-aachen.de} -- BAf"oG-Amt Aachen
\item \url{www.studis-online.de} und \url{www.bafoegrechner.de} -- aktuelle Infos rund ums BAf"oG
\end{aufzaehlung}}
\vspace*{1ex}

Der AStA bietet auch eine pers"onliche Beratung an. Die genauen Termine findest Du in der unten stehenden Kontaktbox oder auf unserer Webseite unter \url{www.asta.rwth-aachen.de/beratung}. Eine vorherige Terminabsprache ist nicht notwendig. Die Beratung ist kostenlos und erfolgt selbstverst"andlich vertraulich.
\end{artikel}
