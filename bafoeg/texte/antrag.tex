\begin{artikel}{Antragstellung in letzter Minute}
\vspace*{-1ex}%Die frage schlucken

\frage{Wann sollte ich meinen Erstantrag abgeben?}
Am besten ca. drei Monate vor Beginn des ersten Semesters. Damit stellst du sicher, dass "uber deinen Antrag rechzeitig entschieden wird um zum Anfang des ersten Semesters F"orderung zu erhalten.

\frage{Wann muss ich meinen Erstantrag abgeben?}
Kurz gesagt: Formell bis sp"atestens Ende des ersten Studienmonats. Allerdings ist davon stark abzuraten, schlie"slich kann die Bearbeitung Deines Antrags eine ganze Weile dauern. Du kannst auch noch nach Ende des ersten Studienmonats einen Antrag stellen hast aber dann keinen Anspruch mehr auf eine Nachzahlung f"ur die bereits verstrichenen Monate.

\frage{Wie lange dauert die Entscheidung "uber meinen Antrag im Normalfall?}
Hast Du Deinen Antrag fr"uhzeitig abgegeben (ca. 3 Monate vor Beginn des ersten Semesters), sollte sp"atestens zum Semesterbeginn ein Bescheid in deinem Briefkasten sein. Vorausgesetzt nat"urlich, Du hast auch alle Nachweise fr"uhzeitig eingereicht. Falls Du zwei Wochen danach noch keinen Bescheid hast, solltest Du Dich beim BAf"oG-Amt melden. Gibst Du Deinen Antrag erst wenige Wochen vor Semesterbeginn ab, musst Du damit rechnen eventuell mehrere Monate warten zu m"ussen. Du erh"alst dann eine Nachzahlung. Das BAf"oG-Amt gibt sich zwar bei Erstantr"agen alle M"uhe, schnell zu bescheiden, wenn aber viele Antr"age sehr kurzfristig gestellt werden, dauert es nat"urlich recht lange. 

\frage{Ich werde den Erstantrag samt Nachweisen nicht rechtzeitig abgeben k"onnen. Was muss ich beachten? }
Falls Du es nicht mehr bis Ende des ersten Studienmonats schaffst: 
Stell in jedem Fall einen formlosen Antrag beim BAf"oG-Amt, um Deinen Anspruch ab Studienbeginn nicht zu verlieren. Es reicht ein Brief mit dem Inhalt: \textit{"`Hiermit beantrage ich, \textit{\{Vorname Nachname\}}, Leistungen nach dem BAf"oG."'}. Kannst Du auch die Frist zur Abgabe der restlichen Unterlagen nicht einhalten, die Dir vom BAf"oG-Amt auf Deinen "`Kurzantrag"' hin gesetzt wurde, sprich dies vorher mit dem Amt ab.
\end{artikel}
