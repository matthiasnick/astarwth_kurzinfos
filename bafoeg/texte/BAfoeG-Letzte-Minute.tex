\begin{artikel}{Diese Textfile wurde geTeXed. Diese Datei wird nicht mehr aktualisiert und hat reinen Archivcharakter. }

BAf"oG-Kurzinfo

Probleme mit dem Bescheid

F: Nach dem Bescheid, den ich erhalten habe, bekomme ich sehr wenig/kein BAf"oG, weil meine Eltern zu viel Einkommen haben. Meine Eltern sind aber nicht bereit, mich zu unterst"utzen, obwohl sie dies finanziell k"onnten. Was kann ich machen?
A: Kannst Du dem Amt f"ur Ausbildungsf"orderung glaubhaft machen, dass Deine Eltern nicht bereit sind, den Anteil zu "ubernehmen, der Dir von Deinem Bedarf abgezogen wurde (sofern er sich auf das Einkommen Deiner Eltern bezieht), ist es m"oglich, f"ur die erste Zeit BAf"oG unter Vorbehalt zu erhalten. Dabei gehen Deine Unterhaltsanspr"uche auf das Land NRW "uber, welches sich widerum bem"uhen wird, das Geld von Deinen Eltern einzutreiben. Dieses Verfahren ist meistens mit der Anh"orung Deiner Eltern verbunden und hat f"ur sie auch rechtliche Konsequenzen. Du solltest nat"urlich alles versuchen, um dieses Problem mit Deinen Eltern anderweitig zu kl"aren. In schwierigeren F"allen hilft es vielleicht, Sie auf obige Regelung hinzuweisen. 

F: Meine Eltern haben vor ein paar Jahren gut verdient. Das hat sich leider jetzt ge"andert und sie k"onnen mich finanziell nicht unterst"utzen. Das BAf"oG-Amt geht aber von ihrem Einkommen vor zwei Jahren aus, wodurch ich jetzt keine oder wenig F"orderung erhalte. Kann ich irgendwas tun?
A: Ja, Du kannst in dieser Situation einen sogenannten Aktualisierungsantrag stellen. Das BAf"oG-Amt prueft dann ob es zu diesem Zeitpunkt finanziell f"ur dich von Vorteil w"are neu zu bescheiden und vom aktuellen Einkommen Deiner Eltern auszugehen. Du solltest Dich vorher aber in jedem Fall beraten lassen, da ein solcher Antrag f"ur Dich mit finanziellen und rechtlichen Risiken verbunden ist. Du kannst beispielsweise spaeter nicht mehr verlangen doch von dem Einkommen vor zwei Jahren auszugehen, falls einer deiner Eltern pl"otzlich eine lukrative Stelle annimmt. Meist musst du in einem solchen Fall die F"orderung kurzfristig zur"uckzahlen.

Hochschul- und Fachwechsel

F: Ich habe bereits an einer anderen Hochschule dasselbe Fach studiert und wechsle zum Anfang des n"achsten Semesters an die RWTH. Kann ich weiter BAf"oG bekommen?
A: Du bekommst im Allgemeinen weiter BAf"oG. Nat"urlich musst Du dazu wie immer einen Folgeantrag stellen, falls Dein Bewilligungszeitraum abl"auft. Au"serdem wird in Bezug auf den "`Leistungsnachweis"', den Du zum 5. Fachsemester erbracht haben musst, keine R"ucksicht auf Hochschulwechsel genommen. Da Du mit Sicherheit Reibungsverluste hast, ist es sinnvoll, die Studienberatung in Anspruch zu nehmen, um zu kl"aren, wie Du den Nachweis rechtzeitig erbringen kannst. 

F: Zum Wechsel an die RWTH habe ich auch mein Studienfach gewechselt. Was muss ich beachten, um weiter BAf"oG zu bekommen?
A: Ein kompliziertes Thema, welches wir hier nur kurz anreissen k"onnen. Ein Fachwechsel ohne Verlust Deines BAf"oG-Anspruchs ist im Allgemeinen nur bis zum Ende des dritten Semesters m"oglich und muss begr"undet werden. Lass Dich dazu beraten; insbesondere wenn Du weiterhin auf Magister studierst, kann es sein, dass BAf"oG-rechtlich "uberhaupt kein Fachwechsel vorliegt.

F: Ich denke der Bescheid ist fehlerhaft. Was muss ich beachten?
A: Versuch Deine F"ordersumme nochmals mit dem BAf"oG-Rechner nachzurechnen (siehe Linkliste). Bist Du immernoch unsicher, dann lass dich beraten. Wenn sich das Problem nicht kurzfristig klaeren laesst, solltest du schriftlich Widerspruch gegen den Bescheid bei Deinem BAfoeG-Amt einlegen. Auf dem Bescheid findest du eine Rechtsbehelfsbelehrung in der dir eine Frist zum einreichen eines Widerspruchs gesetzt wird. Bisher betraegt diese Frist einen Monat nach Bekanntgabe. Sie ist unbedingt einzuhalten.

Die 5 h"aufigsten Fehler beim Erstantrag

1. Den Antrag zu sp"at abgeben.
2. Sich nicht erkundigen, ob von Verwandten Geld auf Deinen Namen angelegt wurde, von dem Du nichts wei"st. (Dieses wird in jedem Fall Dir zugerechnet und gilt bei einem etwaigen Datenabgleich als nicht angegebenes Verm"ogen.)
3. Den ausgefuellten Erstantrag mitten im ersten Semester abgeben, ohne zum Anfang des Semesters einen formlosen Antrag gestellt zu haben. (Du verlierst Deine Anspr"uche vom Anfang des Semesters bis zum Eingang eines Antrags).
4. Sich bei kritischen Fragen nicht beraten lassen. Insbesondere bei der Einkommensaktualisierung.
5. Bei einem "eigenartigen" Bescheid keine Beratung einholen. (Nach einem Monat kannst Du formal gesehen nicht mehr widersprechen und der Bescheid ist f"ur ein Jahr f"ur Dich rechtsverbindlich.). 

Ansprechpersonen bei Problemen

AStA der RWTH Aachen
Turmstra"se 3
52072 Aachen
Tel.: 0241 80-93792
\url{http://www.asta.rwth-aachen.de}
BAf"oG-Beratung eMail: \email{bafoeg@asta.rwth-aachen.de}

Amt f"ur Ausbildungsf"orderung
Turmstra"se 3
52072 Aachen
\url{http://www.studentenwerk-aachen.de}

Linkliste/Weitere Informationen

\url{http://www.bafoeg.bmbf.de} - Bundesministerium f"ur Bildung und Forschung, Informationen und BAf"oG-Rechner
\url{http://www.studentenwerk-aachen.de} - BAf"oG-Amt
\url{http://www.asta.rwth-aachen.de} - Webseiten des AStA mit aktuellen Infos und Sprechzeiten der BAf"oG-Beratung
\url{http://www.studis-online.de} und \url{http://www.bafoegrechner.de} - aktuelle Infos rund ums BAf"oG

\autor{!!!! AUTOR EINFüGEN !!!!}
\end{artikel}
