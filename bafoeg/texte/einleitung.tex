\begin{artikel}{BAf"oG -- alles rund um die Antragstellung}
\begin{window}[5,r,\fcolorbox{black}{light}{\parbox{.4\linewidth}{\textbf{\LARGE Datenabgleich}
\vspace*{1ex}


Wer dem BAf"oG-Amt Verm"ogen verschweigt, macht sich strafbar. Durch verschiedene Gesetze ist das BAf"oG-Amt seit kurzem in der Lage Bankdaten einzusehen. Viele Studierende haben sich auf diesem Wege eine Vorstrafe eingehandelt und mussten hohe Betr"age zur"uckzahlen.

\textbf{Deshalb: Vorsicht!}
}},]
Neben der Entscheidung, welches Fach du an welcher Hochschule studieren m"ochtest, stellt sich zum Studienanfang nat"urlich auch die Frage, der Studienfinanzierung. Dabei geht es im wesentlichen um das Aufbringen der allgemeinen Lebenshaltungskosten, die sp"atestens jetzt, wo Du hoffentlich eine eigene Wohnung in Aachen gefunden hast, auch auf Dich zukommen. F"ur die Menschen, die weder wohlhabende Eltern haben, noch ein Fach studieren, welches genug Zeit l"a"st, um nebenher zu arbeiten, gibt es die Bundesausbildungsf"orderung -- kurz auch BAf"oG genannt (von BundesAusbildungsf"orderungsGesetz).

Um Dich bei der BAf"oG-Beantragung zu unterst"utzen, gibt der AStA eine Info-Brosch"ure heraus, in der alles von der Antragstellung bis zu verschiedensten Ausnahmen und Sonderregelungen erkl"art wird. Dieses Info-Blatt soll daher nur einen "Uberblick und Antworten auf die dr"angendsten Fragen geben, die sich vielen Erstsemestern stellen.

Viele Menschen stellen erst kurz vor Studienbeginn einen BAf"oG-Antrag. Gerade auch f"ur diese Letzte-Minute-AntragstellerInnen haben wir die folgenden Informationen zusammengestellt.
Allerdings ist auch hier empfehlenswert sich gr"undlich zu informieren. Wenn zeitlich "uberhaupt nichts mehr geht, kannst du mit einem formlosen, fristwahrenden Antrag (s.S. 2) erreichen, da"s
die dir zustehende F"orderung nachgezahlt wird. Auf der R"uckseite findest du eine Aufstellung der wichtigsten Anlaufstellen, die zu der Antragstellung, dem BAf"oG-Bescheid oder der R"uckzahlung informieren oder beraten k"onnen. In jedem Fall kann die die BAf"oG-Beratung des AStAs weiterhelfen.
\end{window}
\end{artikel}
