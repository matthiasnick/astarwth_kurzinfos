\begin{artikel}{Hochschul- und Fachwechsel}
\vspace*{-1ex} 

\frage{Ich habe bereits an einer anderen Hochschule dasselbe Fach studiert und wechsle zum Anfang des n"achsten Semesters an die RWTH. Kann ich weiter BAf"oG bekommen?}
Du bekommst im Allgemeinen weiter BAf"oG. Nat"urlich musst Du dazu wie immer einen Folgeantrag stellen, falls Dein Bewilligungszeitraum abl"auft. Au"serdem wird in Bezug auf den "`Leistungsnachweis"', den Du zum 5. Fachsemester erbracht haben musst, keine R"ucksicht auf Hochschulwechsel genommen. Da Du mit Sicherheit Reibungsverluste hast, ist es sinnvoll, die Studienberatung in Anspruch zu nehmen, um zu kl"aren, wie Du den Nachweis rechtzeitig erbringen kannst. 

\frage{Zum Wechsel an die RWTH habe ich auch mein Studienfach/Nebenfach gewechselt. Was muss ich beachten, um weiter BAf"oG zu bekommen?}

Ein kompliziertes Thema, welches wir hier nur kurz anreissen k"onnen. Ein Fachwechsel ohne Verlust Deines BAf"oG-Anspruchs ist im Allgemeinen nur bis zum Ende des dritten Semesters m"oglich und muss begr"undet werden. Lass Dich dazu beraten; insbesondere wenn Du weiterhin auf Magister studierst, kann es sein, dass BAf"oG-rechtlich "uberhaupt kein Fachwechsel vorliegt.

\frage{Ich denke, der Bescheid ist fehlerhaft. Was muss ich beachten?}
Versuche Deine F"ordersumme nochmals mit dem BAf"oG-Rechner nachzurechnen (siehe Linkliste). Bist Du immer noch unsicher, dann lass Dich beraten. Wenn sich das Problem nicht kurzfristig kl"aren l"asst, solltest du schriftlich Widerspruch gegen den Bescheid bei Deinem BAf"oG-Amt einlegen. Auf dem Bescheid findest du eine Rechtsbehelfsbelehrung in der Dir eine Frist zum Einreichen eines Widerspruchs gesetzt wird. Bisher betr"agt diese Frist einen Monat nach Bekanntgabe. Sie ist unbedingt einzuhalten.
\end{artikel}
