\begin{artikel}{Der Mietvertrag}
Die meisten Vermieter/-innen legen Dir einen Standardmietvertrag (z.B. vom Haus- und Grundbesitzerverein) vor. Bei eigenen oder modifizierten Mietvertr�gen ist jedoch auf jeden Fall Vorsicht geboten. Den Vertrag ganz genau zu lesen, sollte sowieso selbstverst�ndlich sein. Ist Dir die Bedeutung von Teilen des Vertrages unklar, solltest Du fachkundigen Rat einholen (z.B. in der Wohnberatung des AStA).

Unterschreibe den Vertrag auf jeden Fall erst, wenn sichergestellt ist, dass Dir nicht irgendwelche Vertragsklauseln zum Nachteil werden 
k�nnten!

\textbf{\large Vorsicht vor:}
\begin{aufzaehlung}
\item Staffelmieten (Mieterh�hungen sind bereits im Mietvertrag festgeschrieben)
\item Zeitmietvertr�gen (hier bist Du f�r einen festen Zeitraum an den Vertrag gebunden und kannst vorher quasi nicht k�ndigen)
\item �bernahme von Wohnungsinventar (Pass auf, dass der/die Vormieter/-in nicht seinen Sperrm�ll an Dich verscherbelt)
\end{aufzaehlung}
\textbf{\large Wichtig:}
\begin{aufzaehlung}
\item Alle Vereinbarungen und Absprachen mit dem/der Vermieter/-in schriftlich festhalten. M�ndliche Absprachen sind rechtlich zwar auch bindend,
aber im Zweifelsfall deutlich schwerer nachzuweisen und einzufordern.
\item Bei der Wohnungs�bergabe ein �bergabeprotokoll anfertigen. Darin alle sichtbaren M�ngel (Feuchtigkeit, Flecken im Teppich oder auf der Tapete usw.)  und �bergabezustand (frisch renoviert?, wenn ja wann?) schriftlich festhalten.
\item Nebenkostenabrechnung des Vorjahres vorlegen lassen, um nicht von hohen Nachzahlungen �berrascht zu werden. Der Vorauszahlungsbetrag 
kann unter Umst�nden sehr weit von den tats�chlichen Nebenkosten abweichen.
\item Genau notieren, welche Sch�den (falls vorhanden) von Vermieter/-in bis wann beseitigt werden. Die Frist hier genau vermerken und 
ihre  Einhaltung im Zweifelsfall auch einfordern.
\item F�r alle F�lle jemanden mitnehmen, der als Zeuge aussagen kann.
\end{aufzaehlung}
\end{artikel}
