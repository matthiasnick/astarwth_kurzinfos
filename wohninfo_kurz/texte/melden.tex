\begin{artikel}{Einwohnermeldeamt}

Hattest Du Erfolg bei der Wohnungssuche, musst Du Dich innerhalb von 14 Tagen nach Einzugsdatum (Mietvertrag) beim 
Einwohnermeldeamt 
melden, da Du sonst mit einem Bu�geld belegt werden kannst. Mitbringen musst Du dazu Deinen Personalausweis und 
den Mietvertrag. 
Es gibt zwei M�glichkeiten, Deinen Wohnsitz anzumelden: als Erst- oder als Zweitwohnsitz. 
Nach dem Meldegesetz musst Du Dich dort mit Erstwohnsitz anmelden, wo Du Deinen Lebensmittelpunkt hast. Dazu zwingen 
kann Dich jedoch Keiner, wenn Du glaubhaft darlegst, dass Du Dich quantitativ mehr am Heimatort aufh�ltst.

Die Anmeldung als Erstwohnsitz f�hrt erst einmal zu Zusatzkosten, da Du eine eigene Hausratversicherung abschlie�en 
solltest. Au�erdem kannst Du in einigen F�llen aus der Familienhaftpflichtversicherung herausfallen (in beiden F�llen im 
Versicherungsvertrag nachschauen!). Falls Du Deinen Ersatzdienst bei der Feuerwehr, beim THW o.�. ableistest, zieht ein 
Wechsel des Erstwohnsitzes au�erdem einen Wechsel Deines Arbeitsplatzes nach sich. Dein Auto musst Du ggf. in Aachen anmelden. AnwohnerInnenparkausweise werden nur an Menschen mit Erstwohnsitz in Aachen vergeben. Das gleiche gilt f�r Wohnberechtigungsscheine. Viele 
Beh�rdeng�nge k�nnen nur am Erstwohnsitz erledigt werden. Hinzu kommt, dass Du nur an Deinem Erstwohnsitz bei Kommunal- und Landtagswahlen wahlberechtigt bist. 

Von Menschen mit einem Zweitwohnsitz in Aachen wird eine Zweitwohnsitzsteuer in H�he von $10\%$ der Nettokaltmiete erhoben. 
Der Anrechnungszeitraum f�r die Steuer beginnt mit dem 
ersten Tag des Folgemonats nach Anmeldung des Zweitwohnsitzes und endet mit dem letzten Tag des Monats, in dem die 
Wohnung aufgegeben wird. Wird Dir die Wohnung aus irgendeinem Grund zu einem Preis unterhalb des Orts�blichen 
�berlassen, wird der Oberwert der Miete aus dem jeweils g�ltigen Mietspiegel zugrunde gelegt. Gemeinschaftlich genutzte 
R�ume werden anteilig auf die Fl�che der Wohnung aufgeschlagen (wichtig bei Wohngemeinschaften!). Ausnahmeregelungen f�r 
finanziell schw�chere Menschen gibt es leider nicht. 
\end{artikel}
